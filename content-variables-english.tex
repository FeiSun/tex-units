% !TeX spellcheck = en_GB

\providecommand{\TITLE}{Lenght units in \TeX}

\providecommand{\INTRO}{%
   The given conversions my suffer form rounding errors and are only for guidance.
   
   The shown rules are \SI{1}{\mm} thick. Please note that the very short rules
   may not be rendered correctly on you screen (or printer) and appear longer
   than they are.
}

\providecommand{\PRIMARYUNITS}{%
   \TestUnit
      {Scaled Point}
      {The scaled point is defined as 1/\num{65536} points.}
      [This is the smallest unit \TeX\ uses.]
      {sp}
   \TestUnit{(PostScript-)Point}{The point is defined as 1/\num{72.27} inch.}{pt}
   \TestUnit
      {Big Point (DTP point)}
      {The big point is defined as 1/\num{72} Inch.}
      [Word, InDesign and other DTP applications use this definition for points.]
      {bp}
   \TestUnit{Didot Point}{An old unit used by European printers}{dd}
   \TestUnit{Millimetre}{SI unit}{mm}
   \TestUnit{Pica}{One pica equals twelve points.}{pc}
   \TestUnit{Cicero}{One Cicero equals twelve Didot points.}{cc}
   \TestUnit{Centimetre}{SI unit}{cm}
   \TestUnit{Inch}{One inch equals \num{2.54} centimetres.}{in}
}

\providecommand{\SECONDARYUNITS}{%
   \TestUnit*
      {Em}
      {The size of an em depends on the current font family and size. It is about the width
         of a capital M and its value is defined in the font file.}
      [This unit should be used for all horizontal distances, that should change with font size,
         e.g. the paragraph indention.]
      {em}
   \TestUnit*
      {x Height}
      {The x height depends on the current font family and size. It is about the height
         of a lower x and its value is defined in the font file.}
      [This unit should be used for all vertical distances, that should change with font size.]
      {ex}
   \TestUnit*
      {Math Unit}
      {This unit equals approx. 1/18 em of the math font family.}
      [It can only be used for spacing in math mode.]
      {mu}
}

\providecommand{\INFOTEXT}{%
   Copyright © 2016, Tobias Weh (\href{http://tobiw.de/en}{tobiw.de/en})\\
   Source code available at \url{https://github.com/tweh/tex-units}.
}