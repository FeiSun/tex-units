% !TeX spellcheck = en_GB

\providecommand{\TITLE}{Length units in \TeX}

\providecommand{\INTRO}{%
   The given conversions may suffer form rounding errors and are only for guidance.
   
   The shown rules are \SI{1}{\mm} thick. Please note that the very short rules
   may not be rendered correctly on your screen (or printer) and appear longer
   than they are.
}

\providecommand{\ABSOLUTEUNITSHEADLINE}{Absolute Units}

\providecommand{\ABSOLUTEUNITSCONTENT}{%
   \TestUnit
      {Scaled Point}
      {The scaled point is defined as 1/\num{65536} points.}
      [This is the smallest unit \TeX\ uses.]
      {sp}
   \TestUnit{Point}{The point is defined as 1/\num{72.27} inch.}{pt}
   \TestUnit
      {Big Point (DTP or PostScript point)}
      {The big point is defined as 1/\num{72} Inch.}
      [Word, InDesign and other DTP applications use this definition for points.]
      {bp}
   \TestUnit{Didot Point}{An old unit used by European printers}{dd}
   \TestUnit{Millimeter}{SI unit}{mm}
   \TestUnit{Pica}{One pica equals twelve points.}{pc}
   \TestUnit{Cicero}{One Cicero equals twelve Didot points.}{cc}
   \TestUnit{Centimeter}{SI unit}{cm}
   \TestUnit{Inch}{One inch equals \num{2.54} centimeters.}{in}
}

\providecommand{\RELATIVEUNITSHEADLINE}{Relative Units}

\providecommand{\RELATIVEUNITSPRETEXT}{%
   These units depend on the currently active font size.
   
   For more details about em and ex see: \url{http://tex.stackexchange.com/q/4239/4918}
}

\providecommand{\RELATIVEUNITSCONTENT}{%
   \TestUnit*
      {Em}
      {Traditionally, an em was defined as the width of a capital M or to be equal to
         the font size, but today the actual value is defined in the font file.}
      [This unit should be used for all horizontal distances that should change relative
         to the font size; the paragraph indention for instance.]
      {em}
   \TestUnit*
      {Ex (x Height)}
      {Traditionally, an ex was defined as the height of a lower case x, but today
         the actual value is defined in the font file.}
      [This unit should be used for all vertical distances that should change relative
         to the font size.]
      {ex}
   \TestUnit*
      {Math Unit}
      {This unit equals approx 1/18 em of the math font family.}
      [It can only be used for spacing in math mode.]
      {mu}
}

\providecommand{\INFOTEXT}{%
   Copyright \raisebox{-0.2ex}{©} 2016, Tobias Weh (\href{http://tobiw.de/en}{tobiw.de/en})\\
   Source code available at: \url{https://github.com/tweh/tex-units}
}