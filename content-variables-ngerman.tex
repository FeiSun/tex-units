
\newcommand{\TITLE}{Lägeneinheiten in \TeX}

\newcommand{\INTRO}{%
   Die Umrechnungen weisen teilweise Rundungsfehler auf und dienen daher nur als
   Orientierung.
   
   Die gezeigten Linien sind immer \SI{1}{\mm} dick. Bitte beachten sie auch,
   dass die besonders kurzen Linien auf Ihrem Ausgabegerät unter Umständen nicht
   genau dargestellt werden können und deswegen dicker erscheinen.
}

\newcommand{\ABSOLUTEUNITSHEADLINE}{Absolute Einheiten}

\newcommand{\ABSOLUTEUNITSCONTENT}{%
   \TestUnit
      {Skalierter Punkt (\textit{Scaled Point})}
      {Ein skalierter Punkt entspricht 1/\num{65536} Punkt.}
      [Der skalierte Punkt ist die kleinste Einheit, auf die
         \TeX\ alles zurückführt.]
      {sp}
   \TestUnit{Punkt}{Ein Punkt entspricht 1/\num{72.27} Zoll.}{pt}
   \TestUnit
      {DTP- oder PostScript-Punkt (\textit{Big Point})}
      {Ein DTP-Punkt entspricht 1/\num{72} Zoll.}
      [Word, InDesign und andere DTP-Programme verwenden diese
         Definition für den Punkt.]
      {bp}
   \TestUnit{Didot-Punkt}{Alte Einheit der europäischen Drucker}{dd}
   \TestUnit{Millimeter}{SI-Einheit}{mm}
   \TestUnit{Pica}{Ein Pica sind zwölf Punkt.}{pc}
   \TestUnit{Cicero}{Ein Cicero sind zwölf Didot-Punk.}{cc}
   \TestUnit{Zentimeter}{SI-Einheit}{cm}
   \TestUnit{Zoll (\textit{Inch})}{Ein Zoll entspricht \num{2.54}
      Zentimetern.}{in}
}

\providecommand{\RELATIVEUNITSHEADLINE}{Relative Einheiten}

\providecommand{\RELATIVEUNITSPRETEXT}{%
   Diese Einheiten sind abhängig von der jeweils verwendeten Schriftgröße.
   
   Details über die Einheiten Em und Ex unter: \url{http://tex.stackexchange.com/q/4239/4918}
   (Englisch)
}

\newcommand{\RELATIVEUNITSCONTENT}{%
   \TestUnit*
      {Geviert (\emph{Em})}
      {Traditionell wurde das Geviert als die Breite eines großen M definiert oder gleich
         der Schriftgröße gesetzt. Heutzutage wird der tatsächliche Wert in der Schriftdatei
         definiert.}
      [Diese Einheit sollte für alle horizontalen Abstände, die sich mit der Schriftgröße
         ändern, verwendet werden, z.\,B. den Absatzeinzug.]
      {em}
   \TestUnit*
      {x-Höhe (\emph{Ex})}
      {Diese Einheit bezog sich auf die Höhe eines kleinen x, wird heute aber wie das
         Gevierit in der Schriftdatei definiert.}
      [Diese Einheit sollte für alle vertikalen Abstände, die sich mit der Schriftgröße
         ändern, verwendet werden.]
      {ex}
   \TestUnit*
      {Matheeinheit (\textit{Math Unit})}
      {Eine Matheeinheit entspricht 1/18 Geviert der aktiven Mathesymbolschrift.}
      [Diese Einheit ist nur im Mathemodus verfügbar und kann dort nur für Abstände
         und Unterschneidungen eingesetzt werden.]
      {mu}
}

\newcommand{\INFOTEXT}{%
   Copyright \raisebox{-0.2ex}{©} 2016, Tobias Weh (\href{http://tobiw.de}{tobiw.de})\\
   Quellcode unter: \url{https://github.com/tweh/tex-units}
}