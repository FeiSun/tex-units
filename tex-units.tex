% To be able to process this file with a script to automatically generate the
% different versions, we define two variable with \def when compiling, like in
%    pdflatex "\def\LANGUAGE{english}\def\COLORMODEL{rgb}% To be able to process this file with a script to automatically generate the
% different versions, we define two variable with \def when compiling, like in
%    pdflatex "\def\LANGUAGE{english}\def\COLORMODEL{rgb}% To be able to process this file with a script to automatically generate the
% different versions, we define two variable with \def when compiling, like in
%    pdflatex "\def\LANGUAGE{english}\def\COLORMODEL{rgb}% To be able to process this file with a script to automatically generate the
% different versions, we define two variable with \def when compiling, like in
%    pdflatex "\def\LANGUAGE{english}\def\COLORMODEL{rgb}\input{tex-units}"
% if we compile the usual way instead, we provide these variables:
\providecommand{\LANGUAGE}{ngerman}% language: 'ngerman' or 'english'
\providecommand{\COLORMODEL}{rgb}% color model: 'rgb' or 'gray'
% We use the variable as class options. YOu may replace them with an appropriate
% value.
\documentclass[
   \LANGUAGE,
   \COLORMODEL,
]{scrartcl}

% LOAD DEFINITIONS
\input{header}

% LANGUAGE VARIABLES
% To be able to switch the language easily, we define some variables in different
% language files. If the language is ngerman we load the corresponding file:
\iflanguage{ngerman}{
   \input{content-variables-ngerman}
}{}
% However we load the english variable in every case to make sure to have the
% variables defined even in unkown languages:
\input{content-variables-english}
% inside of this file we use \providecommand to define the variables so we don't
% overwrite the definitions of other languages.

\begin{document}
% CONTENT
{\Huge\bfseries\TITLE\par}

\vspace{\baselineskip}
\INTRO

\begin{multicols}{2}[\section{\ABSOLUTEUNITSHEADLINE}]
   \ABSOLUTEUNITSCONTENT
\end{multicols}

\vspace{0.5\baselineskip}
\begin{multicols}{3}[\section{\RELATIVEUNITSHEADLINE}\RELATIVEUNITSPRETEXT]
   \RELATIVEUNITSCONTENT
\end{multicols}

\vfill
\footnotesize\itshape
\INFOTEXT

\makebox[0pt][l]{\smash{
   \hspace{\RightColShift}%
   \begin{minipage}[t]{\RightColWidth}
      \vspace{-\textheight}
      \vspace{13mm}
      \hfill
      \DrawRules
   \end{minipage}
}}%
\makebox[0pt][l]{\smash{
   \hspace{\RightColShift}%
   \begin{minipage}[t]{\RightColWidth}
      \DrawRule{dd}{ptbarcolor}%
      \DrawRule{bp}{ptbarcolor}%
      \DrawRule{pt}{ptbarcolor}%
   \end{minipage}
}}%

% about em and ex see also http://tex.stackexchange.com/q/4239/4918.
% Donal Knuth wrote in his TeXBook (S. 60):
% > TeX also recognizes two units of measure that are relative rather than absolute;
% > i.e., they depend on the current context:
% >
% > em is the width of a “quad” in the current font;
% > ex is the “x-height” of the current font.
% >
% > Each font defines its own em and ex values. In olden days, an “em” was the width
% > of an ‘M’, but this is no longer true; ems are simply arbitrary units that come
% > with a font, and so are exes. The Computer Modern fonts have the property that
% > an em-dash is one em wide, each of the digits 0 to 9 is half an em wide, and
% > lowercase ‘x’ is one ex high; but these are not hard-and-fast rules for all
% > fonts. The \rm font (cmr10) of plain TeX has 1em = 10pt and 1ex ≈ 4.3pt; the \bf
% > font (cmbx10) has 1em = 11.5pt and 1ex ≈ 4.44pt; and the \tt font (cmtt10) has
% > 1em = 10.5pt and 1ex ≈ 4.3pt. All of these are “10-point” fonts, yet they have
% > different em and ex values. It is generally best to use em for horizontal
% > measurements and ex for vertical measurements that depend on the current font.

\end{document}"
% if we compile the usual way instead, we provide these variables:
\providecommand{\LANGUAGE}{ngerman}% language: 'ngerman' or 'english'
\providecommand{\COLORMODEL}{rgb}% color model: 'rgb' or 'gray'
% We use the variable as class options. YOu may replace them with an appropriate
% value.
\documentclass[
   \LANGUAGE,
   \COLORMODEL,
]{scrartcl}

% LOAD DEFINITIONS

\usepackage[utf8]{inputenc}% Input encoding (DE: http://tobiw.de/tex-faq#kodierung)
\usepackage[T1]{fontenc}% Font encoding (DE: http://tobiw.de/tex-faq#kodierung)

\usepackage{babel}% Language package (DE: http://tobiw.de/tex-faq#babel)

\usepackage[sfdefault]{FiraSans}% Load font
\KOMAoptions{fontsize = 8.5pt}% base font size

\usepackage{geometry}% page layout (DE: http://tobiw.de/tbdm/layout-1)

\pagestyle{empty}% hide page number
                 % (DE: http://tobiw.de/tbdm/layout-2)

\setlength{\parindent}{0pt}% no par indent
                           % (DE: http://tobiw.de/tex-faq#laengen)

\usepackage{siunitx}% package for numbers/units (DE: http://tobiw.de/tbdm/siunitx)
   \sisetup{
      detect-all,% detect all fonts for numbers/units
   }
   \addto\extrasngerman{\sisetup{locale = DE}}% german setup

\usepackage{xcolor}% color package

\usepackage{graphicx}% we nee \rotatebox

\usepackage{xparse}% for LaTeX3 functions and \NewDocumentCommand etc.

\usepackage{multicol}% multi columns

\usepackage{ragged2e}% nun-justified text
   \RaggedRight

\usepackage{metalogo}% correct settings for \TeX logo etc.
   \setlogokern{eX}{-0.02em}
   \setlogodrop{0.4ex}

\usepackage{hyperref}% interactive documents
   \urlstyle{same}% don't change URL font

% DEFINE TEXT LABELS
\defcaptionname{ngerman}{\definitionname}{Definition}
\defcaptionname{ngerman}{\notename}{Anmerkung}
\defcaptionname{ngerman}{\conversionname}{Umrechnung}
\defcaptionname{english}{\definitionname}{Definition}
\defcaptionname{english}{\notename}{Note}
\defcaptionname{english}{\conversionname}{Conversion}
\AtBeginDocument{
   \providecommand{\definitionname}{Definition}
   \providecommand{\notename}{Note}
   \providecommand{\conversionname}{Conversion}
}

\ExplSyntaxOn% switch on LaTeX3 syntax (DE: http://tobiw.de/tbdm/funktionssymbole)

% DEFINE SOME VARIABLES
% page margins
\dim_new:N \c_ut_page_margin_dim
\dim_set:Nn \c_ut_page_margin_dim { 10mm }
% right column width
\dim_new:N \c_ut_right_layout_col_width_dim
\dim_set:Nn \c_ut_right_layout_col_width_dim { 50mm }
\newlength { \RightColWidth }
\setlength { \RightColWidth } { \c_ut_right_layout_col_width_dim }
% layout column sep
\dim_new:N \c_ut_layout_col_sep_dim
\dim_set:Nn \c_ut_layout_col_sep_dim { 5mm }
% layout column sep
\newlength { \RightColShift }
\AtBeginDocument {
   \setlength { \RightColShift }
      { \dim_eval:n { \textwidth + \c_ut_layout_col_sep_dim } }
}
% inner sep of boxes
\dim_new:N \c_ut_inner_sep_dim
\dim_set:Nn \c_ut_inner_sep_dim { 2mm }
% outer sep of boxes
\dim_new:N \c_ut_outer_sep_dim
\dim_set_eq:NN \c_ut_outer_sep_dim \c_ut_inner_sep_dim
% thickness of small example rule (right at box)
\dim_new:N \c_ut_small_rule_width_dim
\dim_set:Nn \c_ut_small_rule_width_dim { 2mm }
% thickness of big example rules
\dim_new:N \c_ut_big_rule_width_dim
\dim_set:Nn \c_ut_big_rule_width_dim { 4mm }
% width of the boxes containing big example rules
\dim_new:N \c_ut_big_rule_box_width_dim
\dim_set:Nn \c_ut_big_rule_box_width_dim { 5.5mm }
% property list for saving the colors/units for the big rules
\prop_new:N \g_ut_big_rules_prop
% number of current primary color box
\int_new:N \g_ut_current_color_int

% DEFINE AND SETUP SOME COLORS
\definecolorset { rgb / gray } { } { } {
   darkred,     0.643,0.000,0.039 / 0.00 ;
   red,         0.702,0.259,0.090 / 0.00 ;
   orange,      0.918,0.506,0.255 / 0.00 ;
   yellow,      0.949,0.741,0.243 / 0.00 ;
   green,       0.639,0.718,0.251 / 0.00 ;
   teal,        0.000,0.667,0.388 / 0.00 ;
   blue,        0.024,0.318,0.608 / 0.00 ;
   purple,      0.380,0.259,0.514 / 0.00 ;
   darkpurple,  0.325,0.012,0.416 / 0.00 ;
   maintext,    0.000,0.000,0.000 / 0.00 ;
   primboxtext, 0.000,0.000,0.000 / 0.00 ;
   secboxtext,  0.000,0.000,0.000 / 0.00 ;
   secboxfill,  0.000,0.000,0.000 / 0.00 ;
   ptbarcolor,  0.350,0.350,0.350 / 0.35
}
% set main text color
\color { maintext }
% sequence variable to save the colors for the primary boxes
\seq_new:N \c_ut_colors_seq
\seq_set_from_clist:Nn \c_ut_colors_seq {
   darkred,
   red,
   orange,
   yellow,
   green,
   teal,
   blue,
   purple,
   darkpurple
}

% SOME LAYOUT SETTINGS
% page layout
\geometry{
   a4paper,
   margin = \c_ut_page_margin_dim,
   right = \dim_eval:n {
      \c_ut_layout_col_sep_dim
      + \c_ut_right_layout_col_width_dim
      + \c_ut_page_margin_dim
   },
   marginparsep = \c_ut_layout_col_sep_dim,
   marginparwidth = \c_ut_right_layout_col_width_dim,
%   showframe,
}
% space between columns
\setlength { \columnsep } { \c_ut_outer_sep_dim }
% inner spacing of \colorbox
\setlength { \fboxsep } { \c_zero_dim }
% sectioning font should use 'maintextcolor' (. = current color)
%\setkomafont { minisec } { \color { l_ut_fill_color } }
\newkomafont { label } { \sffamily \bfseries \tiny \color { l_ut_fill_color } }

% GENERATE SOME MACRO VARIANTS
\cs_generate_variant:Nn \seq_item:Nn { NN }
\cs_generate_variant:Nn \prop_gput:Nnn { Nxx }

% DEFINE MACROS
% internal macro to print the small labels
\cs_new:Npn \ut_label:n #1 {
   \par \vspace { 3mm }
   { \usekomafont { label } #1 \par }
}
% internal macro to draw the big rules
\cs_new:Npn \ut_make_big_rule:nn #1#2 {
   \makebox [ \c_ut_big_rule_box_width_dim ] {
      \raisebox{ 1em } { \rotatebox { 90 } { 10\,#1 } }
   }
   \hspace  { -\c_ut_big_rule_box_width_dim }
   \makebox [ \c_ut_big_rule_box_width_dim ] { \color { #2 } \rule [ -10#1 ]
      { \c_ut_big_rule_width_dim } { 10#1 } }
}
% user macro to genarate boxes
\NewDocumentCommand { \TestUnit }{ s m m o m } { {
   \int_gincr:N \g_ut_current_color_int
   \IfBooleanTF { #1 } {
      \colorlet { l_ut_text_color } { secboxtext }
      \colorlet { l_ut_fill_color } { secboxfill }
      \dim_set:Nn \l_tmpb_dim {
         \linewidth - 2\c_ut_inner_sep_dim
      }
   } {
      \colorlet { l_ut_text_color } { primboxtext }
      \colorlet { l_ut_fill_color }
         { \seq_item:NN \c_ut_colors_seq \g_ut_current_color_int }
      \prop_gput:Nxx \g_ut_big_rules_prop { #5 }
         { \seq_item:NN \c_ut_colors_seq \g_ut_current_color_int }
      \dim_set:Nn \l_tmpb_dim {
         \linewidth
         - 2\c_ut_inner_sep_dim
         - \c_ut_small_rule_width_dim 
      }
   }
   \tl_if_eq:nnTF { #5 } { mu } {
      \hbox_set:Nn \l_tmpa_box { $ \mkern1mu $ }
   } {
      \hbox_set:Nn \l_tmpa_box { \hspace { 1#5 } }
   }
   \dim_set:Nn \l_tmpa_dim { \box_wd:N \l_tmpa_box }
   \par \vspace { 1\c_ut_outer_sep_dim plus 5mm }
   \colorbox { l_ut_fill_color ! 9 } {
      \rule { 0pt } { \c_ut_inner_sep_dim }
      \hspace { \c_ut_inner_sep_dim }
      \begin{minipage} [ t ] { \l_tmpb_dim }
         \color { l_ut_text_color }
         \minisec { #2 }
         \ut_label:n { \definitionname }
         #3 \par
         \IfValueT { #4 } {
            \ut_label:n { \notename } #4 \par
         }
         \IfBooleanF { #1 } {
            \ut_label:n { \conversionname }
               \SI { 1 } { #5 } ~
               \dim_compare:nF { \l_tmpa_dim = 1mm } {
                   = ~ \SI { \dim_to_decimal_in_unit:nn { \l_tmpa_dim } { 1mm } } { mm } ~
               }
               \dim_compare:nF { \l_tmpa_dim = 1pt } {
                   = ~ \SI { \dim_to_decimal:n { \l_tmpa_dim } } { pt } ~
               }
               \dim_compare:nF { \l_tmpa_dim = 1sp } {
                   = ~ \SI { \dim_to_decimal_in_sp:n { \l_tmpa_dim } } { sp }
               }
         }
         \par \vspace{ \c_ut_inner_sep_dim }
      \end{minipage}
      \hspace { \c_ut_inner_sep_dim }
      \IfBooleanF { #1 } {
         { \color { l_ut_fill_color ! 26 } \vrule width \c_ut_small_rule_width_dim }
         \hspace { -\c_ut_small_rule_width_dim }
         \color { l_ut_fill_color }
         \rule
            [ \dim_eval:n { -\l_tmpa_dim + \c_ut_inner_sep_dim } ]
            { \c_ut_small_rule_width_dim }
            { \l_tmpa_dim }
      }
   }% end colorbox
} }
% user macro to draw the big 500 point rules
\NewDocumentCommand { \DrawRule } { m m } {
   \makebox [ \c_ut_big_rule_box_width_dim ] {
      \raisebox { 3.5em } { \rotatebox [ origin = r ] { 90 } { 500\,#1 } }
   }
   \hspace  { -\c_ut_big_rule_box_width_dim }
   \makebox [ \c_ut_big_rule_box_width_dim ] {
      \color { #2 }
      \rule [ 4.7em ] { \c_ut_big_rule_width_dim } { 500#1 }
   }
}
% user macro to draw the 10 unit rules
\NewDocumentCommand{ \DrawRules } {  } {
   \prop_map_function:NN \g_ut_big_rules_prop \ut_make_big_rule:nn
}

\ExplSyntaxOff% switch off LaTeX3 syntax


% LANGUAGE VARIABLES
% To be able to switch the language easily, we define some variables in different
% language files. If the language is ngerman we load the corresponding file:
\iflanguage{ngerman}{
   
\newcommand{\TITLE}{Lägeneinheiten in \TeX}

\newcommand{\INTRO}{%
   Die Umrechnungen weisen teilweise Rundungsfehler auf und dienen daher nur als
   Orientierung.
   
   Die gezeigten Linien sind immer \SI{1}{\mm} dick. Bitte beachten sie auch,
   dass die besonders kurzen Linien auf Ihrem Ausgabegerät unter Umständen nicht
   genau dargestellt werden können und deswegen dicker erscheinen.
}

\newcommand{\ABSOLUTEUNITSHEADLINE}{Absolute Einheiten}

\newcommand{\ABSOLUTEUNITSCONTENT}{%
   \TestUnit
      {Skalierter Punkt (\textit{Scaled Point})}
      {Ein skalierter Punkt entspricht 1/\num{65536} Punkt.}
      [Der skalierte Punkt ist die kleinste Einheit, auf die
         \TeX\ alles zurückführt.]
      {sp}
   \TestUnit{(PostScript-)Punkt}{Ein Punkt entspricht 1/\num{72.27} Zoll.}{pt}
   \TestUnit
      {DTP-Punkt (\textit{Big Point})}
      {Ein DTP-Punkt entspricht 1/\num{72} Zoll.}
      [Word, InDesign und andere DTP-Programme verwenden diese
         Definition für den Punkt.]
      {bp}
   \TestUnit{Didot-Punkt}{Alte Einheit der europäischen Drucker}{dd}
   \TestUnit{Millimeter}{SI-Einheit}{mm}
   \TestUnit{Pica}{Ein Pica sind zwölf Punkt.}{pc}
   \TestUnit{Cicero}{Ein Cicero sind zwölf Didot-Punk.}{cc}
   \TestUnit{Zentimeter}{SI-Einheit}{cm}
   \TestUnit{Zoll (\textit{Inch})}{Ein Zoll entspricht \num{2.54}
      Zentimetern.}{in}
}

\providecommand{\RELATIVEUNITSHEADLINE}{Relative Einheiten}

\providecommand{\RELATIVEUNITSPRETEXT}{%
   Diese Einheiten sind abhängig von der jeweils verwendeten Schriftgröße.
   
   Details über die Einheiten Em und Ex unter: \url{http://tex.stackexchange.com/q/4239/4918}
   (Englisch)
}

\newcommand{\RELATIVEUNITSCONTENT}{%
   \TestUnit*
      {Geviert (\emph{Em})}
      {Traditionell wurde das Geviert als die Breite eines großen M definiert oder gleich
         der Schriftgröße gesetzt. Heutzutage wird der tatsächliche Wert in der Schriftdatei
         definiert.}
      [Diese Einheit sollte für alle horizontalen Abstände, die sich mit der Schriftgröße
         ändern, verwendet werden, z.\,B. den Absatzeinzug.]
      {em}
   \TestUnit*
      {x-Höhe (\emph{Ex})}
      {Diese Einheit bezog sich auf die Höhe eines kleinen x, wird heute aber wie das
         Gevierit in der Schriftdatei definiert.}
      [Diese Einheit sollte für alle vertikalen Abstände, die sich mit der Schriftgröße
         ändern, verwendet werden.]
      {ex}
   \TestUnit*
      {Matheeinheit (\textit{Math Unit})}
      {Eine Matheeinheit entspricht 1/18 Geviert der aktiven Mathesymbolschrift.}
      [Diese Einheit ist nur im Mathemodus verfügbar und kann dort nur für Abstände
         und Unterschneidungen eingesetzt werden.]
      {mu}
}

\newcommand{\INFOTEXT}{%
   Copyright \raisebox{-0.2ex}{©} 2016, Tobias Weh (\href{http://tobiw.de}{tobiw.de})\\
   Quellcode unter: \url{https://github.com/tweh/tex-units}
}
}{}
% However we load the english variable in every case to make sure to have the
% variables defined even in unkown languages:
% !TeX spellcheck = en_GB

\providecommand{\TITLE}{Length units in \TeX}

\providecommand{\INTRO}{%
   The given conversions may suffer form rounding errors and are only for guidance.
   
   The shown rules are \SI{1}{\mm} thick. Please note that the very short rules
   may not be rendered correctly on your screen (or printer) and appear longer
   than they are.
}

\providecommand{\ABSOLUTEUNITSHEADLINE}{Absolute Units}

\providecommand{\ABSOLUTEUNITSCONTENT}{%
   \TestUnit
      {Scaled Point}
      {The scaled point is defined as 1/\num{65536} points.}
      [This is the smallest unit \TeX\ uses.]
      {sp}
   \TestUnit{(PostScript-)Point}{The point is defined as 1/\num{72.27} inch.}{pt}
   \TestUnit
      {Big Point (DTP point)}
      {The big point is defined as 1/\num{72} Inch.}
      [Word, InDesign and other DTP applications use this definition for points.]
      {bp}
   \TestUnit{Didot Point}{An old unit used by European printers}{dd}
   \TestUnit{Millimeter}{SI unit}{mm}
   \TestUnit{Pica}{One pica equals twelve points.}{pc}
   \TestUnit{Cicero}{One Cicero equals twelve Didot points.}{cc}
   \TestUnit{Centimeter}{SI unit}{cm}
   \TestUnit{Inch}{One inch equals \num{2.54} centimeters.}{in}
}

\providecommand{\RELATIVEUNITSHEADLINE}{Relative Units}

\providecommand{\RELATIVEUNITSPRETEXT}{%
   These units depend on the currently active font size.
   
   For more details about em and ex see: \url{http://tex.stackexchange.com/q/4239/4918}
}

\providecommand{\RELATIVEUNITSCONTENT}{%
   \TestUnit*
      {Em}
      {Traditionally, an em was defined as the width of a capital M or to be equal to
         the font size, but today the actual value is defined in the font file.}
      [This unit should be used for all horizontal distances that should change relative
         to the font size; the paragraph indention for instance.]
      {em}
   \TestUnit*
      {Ex (x Height)}
      {Traditionally, an ex was defined as the height of a lower case x, but today
         the actual value is defined in the font file.}
      [This unit should be used for all vertical distances that should change relative
         to the font size.]
      {ex}
   \TestUnit*
      {Math Unit}
      {This unit equals approx 1/18 em of the math font family.}
      [It can only be used for spacing in math mode.]
      {mu}
}

\providecommand{\INFOTEXT}{%
   Copyright \raisebox{-0.2ex}{©} 2016, Tobias Weh (\href{http://tobiw.de/en}{tobiw.de/en})\\
   Source code available at: \url{https://github.com/tweh/tex-units}
}
% inside of this file we use \providecommand to define the variables so we don't
% overwrite the definitions of other languages.

\begin{document}
% CONTENT
{\Huge\bfseries\TITLE\par}

\vspace{\baselineskip}
\INTRO

\begin{multicols}{2}[\section{\ABSOLUTEUNITSHEADLINE}]
   \ABSOLUTEUNITSCONTENT
\end{multicols}

\vspace{0.5\baselineskip}
\begin{multicols}{3}[\section{\RELATIVEUNITSHEADLINE}\RELATIVEUNITSPRETEXT]
   \RELATIVEUNITSCONTENT
\end{multicols}

\vfill
\footnotesize\itshape
\INFOTEXT

\makebox[0pt][l]{\smash{
   \hspace{\RightColShift}%
   \begin{minipage}[t]{\RightColWidth}
      \vspace{-\textheight}
      \vspace{13mm}
      \hfill
      \DrawRules
   \end{minipage}
}}%
\makebox[0pt][l]{\smash{
   \hspace{\RightColShift}%
   \begin{minipage}[t]{\RightColWidth}
      \DrawRule{dd}{ptbarcolor}%
      \DrawRule{bp}{ptbarcolor}%
      \DrawRule{pt}{ptbarcolor}%
   \end{minipage}
}}%

% about em and ex see also http://tex.stackexchange.com/q/4239/4918.
% Donal Knuth wrote in his TeXBook (S. 60):
% > TeX also recognizes two units of measure that are relative rather than absolute;
% > i.e., they depend on the current context:
% >
% > em is the width of a “quad” in the current font;
% > ex is the “x-height” of the current font.
% >
% > Each font defines its own em and ex values. In olden days, an “em” was the width
% > of an ‘M’, but this is no longer true; ems are simply arbitrary units that come
% > with a font, and so are exes. The Computer Modern fonts have the property that
% > an em-dash is one em wide, each of the digits 0 to 9 is half an em wide, and
% > lowercase ‘x’ is one ex high; but these are not hard-and-fast rules for all
% > fonts. The \rm font (cmr10) of plain TeX has 1em = 10pt and 1ex ≈ 4.3pt; the \bf
% > font (cmbx10) has 1em = 11.5pt and 1ex ≈ 4.44pt; and the \tt font (cmtt10) has
% > 1em = 10.5pt and 1ex ≈ 4.3pt. All of these are “10-point” fonts, yet they have
% > different em and ex values. It is generally best to use em for horizontal
% > measurements and ex for vertical measurements that depend on the current font.

\end{document}"
% if we compile the usual way instead, we provide these variables:
\providecommand{\LANGUAGE}{ngerman}% language: 'ngerman' or 'english'
\providecommand{\COLORMODEL}{rgb}% color model: 'rgb' or 'gray'
% We use the variable as class options. YOu may replace them with an appropriate
% value.
\documentclass[
   \LANGUAGE,
   \COLORMODEL,
]{scrartcl}

% LOAD DEFINITIONS

\usepackage[utf8]{inputenc}% Input encoding (DE: http://tobiw.de/tex-faq#kodierung)
\usepackage[T1]{fontenc}% Font encoding (DE: http://tobiw.de/tex-faq#kodierung)

\usepackage{babel}% Language package (DE: http://tobiw.de/tex-faq#babel)

\usepackage[sfdefault]{FiraSans}% Load font
\KOMAoptions{fontsize = 8.5pt}% base font size

\usepackage{geometry}% page layout (DE: http://tobiw.de/tbdm/layout-1)

\pagestyle{empty}% hide page number
                 % (DE: http://tobiw.de/tbdm/layout-2)

\setlength{\parindent}{0pt}% no par indent
                           % (DE: http://tobiw.de/tex-faq#laengen)

\usepackage{siunitx}% package for numbers/units (DE: http://tobiw.de/tbdm/siunitx)
   \sisetup{
      detect-all,% detect all fonts for numbers/units
   }
   \addto\extrasngerman{\sisetup{locale = DE}}% german setup

\usepackage{xcolor}% color package

\usepackage{graphicx}% we nee \rotatebox

\usepackage{xparse}% for LaTeX3 functions and \NewDocumentCommand etc.

\usepackage{multicol}% multi columns

\usepackage{ragged2e}% nun-justified text
   \RaggedRight

\usepackage{metalogo}% correct settings for \TeX logo etc.
   \setlogokern{eX}{-0.02em}
   \setlogodrop{0.4ex}

\usepackage{hyperref}% interactive documents
   \urlstyle{same}% don't change URL font

% DEFINE TEXT LABELS
\defcaptionname{ngerman}{\definitionname}{Definition}
\defcaptionname{ngerman}{\notename}{Anmerkung}
\defcaptionname{ngerman}{\conversionname}{Umrechnung}
\defcaptionname{english}{\definitionname}{Definition}
\defcaptionname{english}{\notename}{Note}
\defcaptionname{english}{\conversionname}{Conversion}
\AtBeginDocument{
   \providecommand{\definitionname}{Definition}
   \providecommand{\notename}{Note}
   \providecommand{\conversionname}{Conversion}
}

\ExplSyntaxOn% switch on LaTeX3 syntax (DE: http://tobiw.de/tbdm/funktionssymbole)

% DEFINE SOME VARIABLES
% page margins
\dim_new:N \c_ut_page_margin_dim
\dim_set:Nn \c_ut_page_margin_dim { 10mm }
% right column width
\dim_new:N \c_ut_right_layout_col_width_dim
\dim_set:Nn \c_ut_right_layout_col_width_dim { 50mm }
\newlength { \RightColWidth }
\setlength { \RightColWidth } { \c_ut_right_layout_col_width_dim }
% layout column sep
\dim_new:N \c_ut_layout_col_sep_dim
\dim_set:Nn \c_ut_layout_col_sep_dim { 5mm }
% layout column sep
\newlength { \RightColShift }
\AtBeginDocument {
   \setlength { \RightColShift }
      { \dim_eval:n { \textwidth + \c_ut_layout_col_sep_dim } }
}
% inner sep of boxes
\dim_new:N \c_ut_inner_sep_dim
\dim_set:Nn \c_ut_inner_sep_dim { 2mm }
% outer sep of boxes
\dim_new:N \c_ut_outer_sep_dim
\dim_set_eq:NN \c_ut_outer_sep_dim \c_ut_inner_sep_dim
% thickness of small example rule (right at box)
\dim_new:N \c_ut_small_rule_width_dim
\dim_set:Nn \c_ut_small_rule_width_dim { 2mm }
% thickness of big example rules
\dim_new:N \c_ut_big_rule_width_dim
\dim_set:Nn \c_ut_big_rule_width_dim { 4mm }
% width of the boxes containing big example rules
\dim_new:N \c_ut_big_rule_box_width_dim
\dim_set:Nn \c_ut_big_rule_box_width_dim { 5.5mm }
% property list for saving the colors/units for the big rules
\prop_new:N \g_ut_big_rules_prop
% number of current primary color box
\int_new:N \g_ut_current_color_int

% DEFINE AND SETUP SOME COLORS
\definecolorset { rgb / gray } { } { } {
   darkred,     0.643,0.000,0.039 / 0.00 ;
   red,         0.702,0.259,0.090 / 0.00 ;
   orange,      0.918,0.506,0.255 / 0.00 ;
   yellow,      0.949,0.741,0.243 / 0.00 ;
   green,       0.639,0.718,0.251 / 0.00 ;
   teal,        0.000,0.667,0.388 / 0.00 ;
   blue,        0.024,0.318,0.608 / 0.00 ;
   purple,      0.380,0.259,0.514 / 0.00 ;
   darkpurple,  0.325,0.012,0.416 / 0.00 ;
   maintext,    0.000,0.000,0.000 / 0.00 ;
   primboxtext, 0.000,0.000,0.000 / 0.00 ;
   secboxtext,  0.000,0.000,0.000 / 0.00 ;
   secboxfill,  0.000,0.000,0.000 / 0.00 ;
   ptbarcolor,  0.350,0.350,0.350 / 0.35
}
% set main text color
\color { maintext }
% sequence variable to save the colors for the primary boxes
\seq_new:N \c_ut_colors_seq
\seq_set_from_clist:Nn \c_ut_colors_seq {
   darkred,
   red,
   orange,
   yellow,
   green,
   teal,
   blue,
   purple,
   darkpurple
}

% SOME LAYOUT SETTINGS
% page layout
\geometry{
   a4paper,
   margin = \c_ut_page_margin_dim,
   right = \dim_eval:n {
      \c_ut_layout_col_sep_dim
      + \c_ut_right_layout_col_width_dim
      + \c_ut_page_margin_dim
   },
   marginparsep = \c_ut_layout_col_sep_dim,
   marginparwidth = \c_ut_right_layout_col_width_dim,
%   showframe,
}
% space between columns
\setlength { \columnsep } { \c_ut_outer_sep_dim }
% inner spacing of \colorbox
\setlength { \fboxsep } { \c_zero_dim }
% sectioning font should use 'maintextcolor' (. = current color)
%\setkomafont { minisec } { \color { l_ut_fill_color } }
\newkomafont { label } { \sffamily \bfseries \tiny \color { l_ut_fill_color } }

% GENERATE SOME MACRO VARIANTS
\cs_generate_variant:Nn \seq_item:Nn { NN }
\cs_generate_variant:Nn \prop_gput:Nnn { Nxx }

% DEFINE MACROS
% internal macro to print the small labels
\cs_new:Npn \ut_label:n #1 {
   \par \vspace { 3mm }
   { \usekomafont { label } #1 \par }
}
% internal macro to draw the big rules
\cs_new:Npn \ut_make_big_rule:nn #1#2 {
   \makebox [ \c_ut_big_rule_box_width_dim ] {
      \raisebox{ 1em } { \rotatebox { 90 } { 10\,#1 } }
   }
   \hspace  { -\c_ut_big_rule_box_width_dim }
   \makebox [ \c_ut_big_rule_box_width_dim ] { \color { #2 } \rule [ -10#1 ]
      { \c_ut_big_rule_width_dim } { 10#1 } }
}
% user macro to genarate boxes
\NewDocumentCommand { \TestUnit }{ s m m o m } { {
   \int_gincr:N \g_ut_current_color_int
   \IfBooleanTF { #1 } {
      \colorlet { l_ut_text_color } { secboxtext }
      \colorlet { l_ut_fill_color } { secboxfill }
      \dim_set:Nn \l_tmpb_dim {
         \linewidth - 2\c_ut_inner_sep_dim
      }
   } {
      \colorlet { l_ut_text_color } { primboxtext }
      \colorlet { l_ut_fill_color }
         { \seq_item:NN \c_ut_colors_seq \g_ut_current_color_int }
      \prop_gput:Nxx \g_ut_big_rules_prop { #5 }
         { \seq_item:NN \c_ut_colors_seq \g_ut_current_color_int }
      \dim_set:Nn \l_tmpb_dim {
         \linewidth
         - 2\c_ut_inner_sep_dim
         - \c_ut_small_rule_width_dim 
      }
   }
   \tl_if_eq:nnTF { #5 } { mu } {
      \hbox_set:Nn \l_tmpa_box { $ \mkern1mu $ }
   } {
      \hbox_set:Nn \l_tmpa_box { \hspace { 1#5 } }
   }
   \dim_set:Nn \l_tmpa_dim { \box_wd:N \l_tmpa_box }
   \par \vspace { 1\c_ut_outer_sep_dim plus 5mm }
   \colorbox { l_ut_fill_color ! 9 } {
      \rule { 0pt } { \c_ut_inner_sep_dim }
      \hspace { \c_ut_inner_sep_dim }
      \begin{minipage} [ t ] { \l_tmpb_dim }
         \color { l_ut_text_color }
         \minisec { #2 }
         \ut_label:n { \definitionname }
         #3 \par
         \IfValueT { #4 } {
            \ut_label:n { \notename } #4 \par
         }
         \IfBooleanF { #1 } {
            \ut_label:n { \conversionname }
               \SI { 1 } { #5 } ~
               \dim_compare:nF { \l_tmpa_dim = 1mm } {
                   = ~ \SI { \dim_to_decimal_in_unit:nn { \l_tmpa_dim } { 1mm } } { mm } ~
               }
               \dim_compare:nF { \l_tmpa_dim = 1pt } {
                   = ~ \SI { \dim_to_decimal:n { \l_tmpa_dim } } { pt } ~
               }
               \dim_compare:nF { \l_tmpa_dim = 1sp } {
                   = ~ \SI { \dim_to_decimal_in_sp:n { \l_tmpa_dim } } { sp }
               }
         }
         \par \vspace{ \c_ut_inner_sep_dim }
      \end{minipage}
      \hspace { \c_ut_inner_sep_dim }
      \IfBooleanF { #1 } {
         { \color { l_ut_fill_color ! 26 } \vrule width \c_ut_small_rule_width_dim }
         \hspace { -\c_ut_small_rule_width_dim }
         \color { l_ut_fill_color }
         \rule
            [ \dim_eval:n { -\l_tmpa_dim + \c_ut_inner_sep_dim } ]
            { \c_ut_small_rule_width_dim }
            { \l_tmpa_dim }
      }
   }% end colorbox
} }
% user macro to draw the big 500 point rules
\NewDocumentCommand { \DrawRule } { m m } {
   \makebox [ \c_ut_big_rule_box_width_dim ] {
      \raisebox { 3.5em } { \rotatebox [ origin = r ] { 90 } { 500\,#1 } }
   }
   \hspace  { -\c_ut_big_rule_box_width_dim }
   \makebox [ \c_ut_big_rule_box_width_dim ] {
      \color { #2 }
      \rule [ 4.7em ] { \c_ut_big_rule_width_dim } { 500#1 }
   }
}
% user macro to draw the 10 unit rules
\NewDocumentCommand{ \DrawRules } {  } {
   \prop_map_function:NN \g_ut_big_rules_prop \ut_make_big_rule:nn
}

\ExplSyntaxOff% switch off LaTeX3 syntax


% LANGUAGE VARIABLES
% To be able to switch the language easily, we define some variables in different
% language files. If the language is ngerman we load the corresponding file:
\iflanguage{ngerman}{
   
\newcommand{\TITLE}{Lägeneinheiten in \TeX}

\newcommand{\INTRO}{%
   Die Umrechnungen weisen teilweise Rundungsfehler auf und dienen daher nur als
   Orientierung.
   
   Die gezeigten Linien sind immer \SI{1}{\mm} dick. Bitte beachten sie auch,
   dass die besonders kurzen Linien auf Ihrem Ausgabegerät unter Umständen nicht
   genau dargestellt werden können und deswegen dicker erscheinen.
}

\newcommand{\ABSOLUTEUNITSHEADLINE}{Absolute Einheiten}

\newcommand{\ABSOLUTEUNITSCONTENT}{%
   \TestUnit
      {Skalierter Punkt (\textit{Scaled Point})}
      {Ein skalierter Punkt entspricht 1/\num{65536} Punkt.}
      [Der skalierte Punkt ist die kleinste Einheit, auf die
         \TeX\ alles zurückführt.]
      {sp}
   \TestUnit{(PostScript-)Punkt}{Ein Punkt entspricht 1/\num{72.27} Zoll.}{pt}
   \TestUnit
      {DTP-Punkt (\textit{Big Point})}
      {Ein DTP-Punkt entspricht 1/\num{72} Zoll.}
      [Word, InDesign und andere DTP-Programme verwenden diese
         Definition für den Punkt.]
      {bp}
   \TestUnit{Didot-Punkt}{Alte Einheit der europäischen Drucker}{dd}
   \TestUnit{Millimeter}{SI-Einheit}{mm}
   \TestUnit{Pica}{Ein Pica sind zwölf Punkt.}{pc}
   \TestUnit{Cicero}{Ein Cicero sind zwölf Didot-Punk.}{cc}
   \TestUnit{Zentimeter}{SI-Einheit}{cm}
   \TestUnit{Zoll (\textit{Inch})}{Ein Zoll entspricht \num{2.54}
      Zentimetern.}{in}
}

\providecommand{\RELATIVEUNITSHEADLINE}{Relative Einheiten}

\providecommand{\RELATIVEUNITSPRETEXT}{%
   Diese Einheiten sind abhängig von der jeweils verwendeten Schriftgröße.
   
   Details über die Einheiten Em und Ex unter: \url{http://tex.stackexchange.com/q/4239/4918}
   (Englisch)
}

\newcommand{\RELATIVEUNITSCONTENT}{%
   \TestUnit*
      {Geviert (\emph{Em})}
      {Traditionell wurde das Geviert als die Breite eines großen M definiert oder gleich
         der Schriftgröße gesetzt. Heutzutage wird der tatsächliche Wert in der Schriftdatei
         definiert.}
      [Diese Einheit sollte für alle horizontalen Abstände, die sich mit der Schriftgröße
         ändern, verwendet werden, z.\,B. den Absatzeinzug.]
      {em}
   \TestUnit*
      {x-Höhe (\emph{Ex})}
      {Diese Einheit bezog sich auf die Höhe eines kleinen x, wird heute aber wie das
         Gevierit in der Schriftdatei definiert.}
      [Diese Einheit sollte für alle vertikalen Abstände, die sich mit der Schriftgröße
         ändern, verwendet werden.]
      {ex}
   \TestUnit*
      {Matheeinheit (\textit{Math Unit})}
      {Eine Matheeinheit entspricht 1/18 Geviert der aktiven Mathesymbolschrift.}
      [Diese Einheit ist nur im Mathemodus verfügbar und kann dort nur für Abstände
         und Unterschneidungen eingesetzt werden.]
      {mu}
}

\newcommand{\INFOTEXT}{%
   Copyright \raisebox{-0.2ex}{©} 2016, Tobias Weh (\href{http://tobiw.de}{tobiw.de})\\
   Quellcode unter: \url{https://github.com/tweh/tex-units}
}
}{}
% However we load the english variable in every case to make sure to have the
% variables defined even in unkown languages:
% !TeX spellcheck = en_GB

\providecommand{\TITLE}{Length units in \TeX}

\providecommand{\INTRO}{%
   The given conversions may suffer form rounding errors and are only for guidance.
   
   The shown rules are \SI{1}{\mm} thick. Please note that the very short rules
   may not be rendered correctly on your screen (or printer) and appear longer
   than they are.
}

\providecommand{\ABSOLUTEUNITSHEADLINE}{Absolute Units}

\providecommand{\ABSOLUTEUNITSCONTENT}{%
   \TestUnit
      {Scaled Point}
      {The scaled point is defined as 1/\num{65536} points.}
      [This is the smallest unit \TeX\ uses.]
      {sp}
   \TestUnit{(PostScript-)Point}{The point is defined as 1/\num{72.27} inch.}{pt}
   \TestUnit
      {Big Point (DTP point)}
      {The big point is defined as 1/\num{72} Inch.}
      [Word, InDesign and other DTP applications use this definition for points.]
      {bp}
   \TestUnit{Didot Point}{An old unit used by European printers}{dd}
   \TestUnit{Millimeter}{SI unit}{mm}
   \TestUnit{Pica}{One pica equals twelve points.}{pc}
   \TestUnit{Cicero}{One Cicero equals twelve Didot points.}{cc}
   \TestUnit{Centimeter}{SI unit}{cm}
   \TestUnit{Inch}{One inch equals \num{2.54} centimeters.}{in}
}

\providecommand{\RELATIVEUNITSHEADLINE}{Relative Units}

\providecommand{\RELATIVEUNITSPRETEXT}{%
   These units depend on the currently active font size.
   
   For more details about em and ex see: \url{http://tex.stackexchange.com/q/4239/4918}
}

\providecommand{\RELATIVEUNITSCONTENT}{%
   \TestUnit*
      {Em}
      {Traditionally, an em was defined as the width of a capital M or to be equal to
         the font size, but today the actual value is defined in the font file.}
      [This unit should be used for all horizontal distances that should change relative
         to the font size; the paragraph indention for instance.]
      {em}
   \TestUnit*
      {Ex (x Height)}
      {Traditionally, an ex was defined as the height of a lower case x, but today
         the actual value is defined in the font file.}
      [This unit should be used for all vertical distances that should change relative
         to the font size.]
      {ex}
   \TestUnit*
      {Math Unit}
      {This unit equals approx 1/18 em of the math font family.}
      [It can only be used for spacing in math mode.]
      {mu}
}

\providecommand{\INFOTEXT}{%
   Copyright \raisebox{-0.2ex}{©} 2016, Tobias Weh (\href{http://tobiw.de/en}{tobiw.de/en})\\
   Source code available at: \url{https://github.com/tweh/tex-units}
}
% inside of this file we use \providecommand to define the variables so we don't
% overwrite the definitions of other languages.

\begin{document}
% CONTENT
{\Huge\bfseries\TITLE\par}

\vspace{\baselineskip}
\INTRO

\begin{multicols}{2}[\section{\ABSOLUTEUNITSHEADLINE}]
   \ABSOLUTEUNITSCONTENT
\end{multicols}

\vspace{0.5\baselineskip}
\begin{multicols}{3}[\section{\RELATIVEUNITSHEADLINE}\RELATIVEUNITSPRETEXT]
   \RELATIVEUNITSCONTENT
\end{multicols}

\vfill
\footnotesize\itshape
\INFOTEXT

\makebox[0pt][l]{\smash{
   \hspace{\RightColShift}%
   \begin{minipage}[t]{\RightColWidth}
      \vspace{-\textheight}
      \vspace{13mm}
      \hfill
      \DrawRules
   \end{minipage}
}}%
\makebox[0pt][l]{\smash{
   \hspace{\RightColShift}%
   \begin{minipage}[t]{\RightColWidth}
      \DrawRule{dd}{ptbarcolor}%
      \DrawRule{bp}{ptbarcolor}%
      \DrawRule{pt}{ptbarcolor}%
   \end{minipage}
}}%

% about em and ex see also http://tex.stackexchange.com/q/4239/4918.
% Donal Knuth wrote in his TeXBook (S. 60):
% > TeX also recognizes two units of measure that are relative rather than absolute;
% > i.e., they depend on the current context:
% >
% > em is the width of a “quad” in the current font;
% > ex is the “x-height” of the current font.
% >
% > Each font defines its own em and ex values. In olden days, an “em” was the width
% > of an ‘M’, but this is no longer true; ems are simply arbitrary units that come
% > with a font, and so are exes. The Computer Modern fonts have the property that
% > an em-dash is one em wide, each of the digits 0 to 9 is half an em wide, and
% > lowercase ‘x’ is one ex high; but these are not hard-and-fast rules for all
% > fonts. The \rm font (cmr10) of plain TeX has 1em = 10pt and 1ex ≈ 4.3pt; the \bf
% > font (cmbx10) has 1em = 11.5pt and 1ex ≈ 4.44pt; and the \tt font (cmtt10) has
% > 1em = 10.5pt and 1ex ≈ 4.3pt. All of these are “10-point” fonts, yet they have
% > different em and ex values. It is generally best to use em for horizontal
% > measurements and ex for vertical measurements that depend on the current font.

\end{document}"
% if we compile the usual way instead, we provide these variables:
\providecommand{\LANGUAGE}{eglish}% language: 'ngerman' or 'english'
\providecommand{\COLORMODEL}{rgb}% color model: 'rgb' or 'gray'
% We use the variable as class options. YOu may replace them with an appropriate
% value.
\documentclass[
   \LANGUAGE,
   \COLORMODEL,
]{scrartcl}

% LOAD DEFINITIONS

\usepackage[utf8]{inputenc}% Input encoding (DE: http://tobiw.de/tex-faq#kodierung)
\usepackage[T1]{fontenc}% Font encoding (DE: http://tobiw.de/tex-faq#kodierung)

\usepackage{babel}% Language package (DE: http://tobiw.de/tex-faq#babel)

\usepackage[sfdefault]{FiraSans}% Load font
\KOMAoptions{fontsize = 8.5pt}% base font size

\usepackage{geometry}% page layout (DE: http://tobiw.de/tbdm/layout-1)

\pagestyle{empty}% hide page number
                 % (DE: http://tobiw.de/tbdm/layout-2)

\setlength{\parindent}{0pt}% no par indent
                           % (DE: http://tobiw.de/tex-faq#laengen)

\usepackage{siunitx}% package for numbers/units (DE: http://tobiw.de/tbdm/siunitx)
   \sisetup{
      detect-all,% detect all fonts for numbers/units
   }
   \addto\extrasngerman{\sisetup{locale = DE}}% german setup

\usepackage{xcolor}% color package

\usepackage{graphicx}% we nee \rotatebox

\usepackage{xparse}% for LaTeX3 functions and \NewDocumentCommand etc.

\usepackage{multicol}% multi columns

\usepackage{ragged2e}% nun-justified text
   \RaggedRight

\usepackage{metalogo}% correct settings for \TeX logo etc.
   \setlogokern{eX}{-0.02em}
   \setlogodrop{0.4ex}

\usepackage{hyperref}% interactive documents
   \urlstyle{same}% don't change URL font

% DEFINE TEXT LABELS
\defcaptionname{ngerman}{\definitionname}{Definition}
\defcaptionname{ngerman}{\notename}{Anmerkung}
\defcaptionname{ngerman}{\conversionname}{Umrechnung}
\defcaptionname{english}{\definitionname}{Definition}
\defcaptionname{english}{\notename}{Note}
\defcaptionname{english}{\conversionname}{Conversion}
\AtBeginDocument{
   \providecommand{\definitionname}{Definition}
   \providecommand{\notename}{Note}
   \providecommand{\conversionname}{Conversion}
}

\ExplSyntaxOn% switch on LaTeX3 syntax (DE: http://tobiw.de/tbdm/funktionssymbole)

% DEFINE SOME VARIABLES
% page margins
\dim_new:N \c_ut_page_margin_dim
\dim_set:Nn \c_ut_page_margin_dim { 10mm }
% right column width
\dim_new:N \c_ut_right_layout_col_width_dim
\dim_set:Nn \c_ut_right_layout_col_width_dim { 50mm }
\newlength { \RightColWidth }
\setlength { \RightColWidth } { \c_ut_right_layout_col_width_dim }
% layout column sep
\dim_new:N \c_ut_layout_col_sep_dim
\dim_set:Nn \c_ut_layout_col_sep_dim { 5mm }
% layout column sep
\newlength { \RightColShift }
\AtBeginDocument {
   \setlength { \RightColShift }
      { \dim_eval:n { \textwidth + \c_ut_layout_col_sep_dim } }
}
% inner sep of boxes
\dim_new:N \c_ut_inner_sep_dim
\dim_set:Nn \c_ut_inner_sep_dim { 2mm }
% outer sep of boxes
\dim_new:N \c_ut_outer_sep_dim
\dim_set_eq:NN \c_ut_outer_sep_dim \c_ut_inner_sep_dim
% thickness of small example rule (right at box)
\dim_new:N \c_ut_small_rule_width_dim
\dim_set:Nn \c_ut_small_rule_width_dim { 2mm }
% thickness of big example rules
\dim_new:N \c_ut_big_rule_width_dim
\dim_set:Nn \c_ut_big_rule_width_dim { 4mm }
% width of the boxes containing big example rules
\dim_new:N \c_ut_big_rule_box_width_dim
\dim_set:Nn \c_ut_big_rule_box_width_dim { 5.5mm }
% property list for saving the colors/units for the big rules
\prop_new:N \g_ut_big_rules_prop
% number of current primary color box
\int_new:N \g_ut_current_color_int

% DEFINE AND SETUP SOME COLORS
\definecolorset { rgb / gray } { } { } {
   darkred,     0.643,0.000,0.039 / 0.00 ;
   red,         0.702,0.259,0.090 / 0.00 ;
   orange,      0.918,0.506,0.255 / 0.00 ;
   yellow,      0.949,0.741,0.243 / 0.00 ;
   green,       0.639,0.718,0.251 / 0.00 ;
   teal,        0.000,0.667,0.388 / 0.00 ;
   blue,        0.024,0.318,0.608 / 0.00 ;
   purple,      0.380,0.259,0.514 / 0.00 ;
   darkpurple,  0.325,0.012,0.416 / 0.00 ;
   maintext,    0.000,0.000,0.000 / 0.00 ;
   primboxtext, 0.000,0.000,0.000 / 0.00 ;
   secboxtext,  0.000,0.000,0.000 / 0.00 ;
   secboxfill,  0.000,0.000,0.000 / 0.00 ;
   ptbarcolor,  0.350,0.350,0.350 / 0.35
}
% set main text color
\color { maintext }
% sequence variable to save the colors for the primary boxes
\seq_new:N \c_ut_colors_seq
\seq_set_from_clist:Nn \c_ut_colors_seq {
   darkred,
   red,
   orange,
   yellow,
   green,
   teal,
   blue,
   purple,
   darkpurple
}

% SOME LAYOUT SETTINGS
% page layout
\geometry{
   a4paper,
   margin = \c_ut_page_margin_dim,
   right = \dim_eval:n {
      \c_ut_layout_col_sep_dim
      + \c_ut_right_layout_col_width_dim
      + \c_ut_page_margin_dim
   },
   marginparsep = \c_ut_layout_col_sep_dim,
   marginparwidth = \c_ut_right_layout_col_width_dim,
%   showframe,
}
% space between columns
\setlength { \columnsep } { \c_ut_outer_sep_dim }
% inner spacing of \colorbox
\setlength { \fboxsep } { \c_zero_dim }
% sectioning font should use 'maintextcolor' (. = current color)
%\setkomafont { minisec } { \color { l_ut_fill_color } }
\newkomafont { label } { \sffamily \bfseries \tiny \color { l_ut_fill_color } }

% GENERATE SOME MACRO VARIANTS
\cs_generate_variant:Nn \seq_item:Nn { NN }
\cs_generate_variant:Nn \prop_gput:Nnn { Nxx }

% DEFINE MACROS
% internal macro to print the small labels
\cs_new:Npn \ut_label:n #1 {
   \par \vspace { 3mm }
   { \usekomafont { label } #1 \par }
}
% internal macro to draw the big rules
\cs_new:Npn \ut_make_big_rule:nn #1#2 {
   \makebox [ \c_ut_big_rule_box_width_dim ] {
      \raisebox{ 1em } { \rotatebox { 90 } { 10\,#1 } }
   }
   \hspace  { -\c_ut_big_rule_box_width_dim }
   \makebox [ \c_ut_big_rule_box_width_dim ] { \color { #2 } \rule [ -10#1 ]
      { \c_ut_big_rule_width_dim } { 10#1 } }
}
% user macro to genarate boxes
\NewDocumentCommand { \TestUnit }{ s m m o m } { {
   \int_gincr:N \g_ut_current_color_int
   \IfBooleanTF { #1 } {
      \colorlet { l_ut_text_color } { secboxtext }
      \colorlet { l_ut_fill_color } { secboxfill }
      \dim_set:Nn \l_tmpb_dim {
         \linewidth - 2\c_ut_inner_sep_dim
      }
   } {
      \colorlet { l_ut_text_color } { primboxtext }
      \colorlet { l_ut_fill_color }
         { \seq_item:NN \c_ut_colors_seq \g_ut_current_color_int }
      \prop_gput:Nxx \g_ut_big_rules_prop { #5 }
         { \seq_item:NN \c_ut_colors_seq \g_ut_current_color_int }
      \dim_set:Nn \l_tmpb_dim {
         \linewidth
         - 2\c_ut_inner_sep_dim
         - \c_ut_small_rule_width_dim 
      }
   }
   \tl_if_eq:nnTF { #5 } { mu } {
      \hbox_set:Nn \l_tmpa_box { $ \mkern1mu $ }
   } {
      \hbox_set:Nn \l_tmpa_box { \hspace { 1#5 } }
   }
   \dim_set:Nn \l_tmpa_dim { \box_wd:N \l_tmpa_box }
   \par \vspace { 1\c_ut_outer_sep_dim plus 5mm }
   \colorbox { l_ut_fill_color ! 9 } {
      \rule { 0pt } { \c_ut_inner_sep_dim }
      \hspace { \c_ut_inner_sep_dim }
      \begin{minipage} [ t ] { \l_tmpb_dim }
         \color { l_ut_text_color }
         \minisec { #2 }
         \ut_label:n { \definitionname }
         #3 \par
         \IfValueT { #4 } {
            \ut_label:n { \notename } #4 \par
         }
         \IfBooleanF { #1 } {
            \ut_label:n { \conversionname }
               \SI { 1 } { #5 } ~
               \dim_compare:nF { \l_tmpa_dim = 1mm } {
                   = ~ \SI { \dim_to_decimal_in_unit:nn { \l_tmpa_dim } { 1mm } } { mm } ~
               }
               \dim_compare:nF { \l_tmpa_dim = 1pt } {
                   = ~ \SI { \dim_to_decimal:n { \l_tmpa_dim } } { pt } ~
               }
               \dim_compare:nF { \l_tmpa_dim = 1sp } {
                   = ~ \SI { \dim_to_decimal_in_sp:n { \l_tmpa_dim } } { sp }
               }
         }
         \par \vspace{ \c_ut_inner_sep_dim }
      \end{minipage}
      \hspace { \c_ut_inner_sep_dim }
      \IfBooleanF { #1 } {
         { \color { l_ut_fill_color ! 26 } \vrule width \c_ut_small_rule_width_dim }
         \hspace { -\c_ut_small_rule_width_dim }
         \color { l_ut_fill_color }
         \rule
            [ \dim_eval:n { -\l_tmpa_dim + \c_ut_inner_sep_dim } ]
            { \c_ut_small_rule_width_dim }
            { \l_tmpa_dim }
      }
   }% end colorbox
} }
% user macro to draw the big 500 point rules
\NewDocumentCommand { \DrawRule } { m m } {
   \makebox [ \c_ut_big_rule_box_width_dim ] {
      \raisebox { 3.5em } { \rotatebox [ origin = r ] { 90 } { 500\,#1 } }
   }
   \hspace  { -\c_ut_big_rule_box_width_dim }
   \makebox [ \c_ut_big_rule_box_width_dim ] {
      \color { #2 }
      \rule [ 4.7em ] { \c_ut_big_rule_width_dim } { 500#1 }
   }
}
% user macro to draw the 10 unit rules
\NewDocumentCommand{ \DrawRules } {  } {
   \prop_map_function:NN \g_ut_big_rules_prop \ut_make_big_rule:nn
}

\ExplSyntaxOff% switch off LaTeX3 syntax


% LANGUAGE VARIABLES
% To be able to switch the language easily, we define some variables in different
% language files. If the language is ngerman we load the corresponding file:
\iflanguage{ngerman}{
   
\newcommand{\TITLE}{Lägeneinheiten in \TeX}

\newcommand{\INTRO}{%
   Die Umrechnungen weisen teilweise Rundungsfehler auf und dienen daher nur als
   Orientierung.
   
   Die gezeigten Linien sind immer \SI{1}{\mm} dick. Bitte beachten sie auch,
   dass die besonders kurzen Linien auf Ihrem Ausgabegerät unter Umständen nicht
   genau dargestellt werden können und deswegen dicker erscheinen.
}

\newcommand{\ABSOLUTEUNITSHEADLINE}{Absolute Einheiten}

\newcommand{\ABSOLUTEUNITSCONTENT}{%
   \TestUnit
      {Skalierter Punkt (\textit{Scaled Point})}
      {Ein skalierter Punkt entspricht 1/\num{65536} Punkt.}
      [Der skalierte Punkt ist die kleinste Einheit, auf die
         \TeX\ alles zurückführt.]
      {sp}
   \TestUnit{(PostScript-)Punkt}{Ein Punkt entspricht 1/\num{72.27} Zoll.}{pt}
   \TestUnit
      {DTP-Punkt (\textit{Big Point})}
      {Ein DTP-Punkt entspricht 1/\num{72} Zoll.}
      [Word, InDesign und andere DTP-Programme verwenden diese
         Definition für den Punkt.]
      {bp}
   \TestUnit{Didot-Punkt}{Alte Einheit der europäischen Drucker}{dd}
   \TestUnit{Millimeter}{SI-Einheit}{mm}
   \TestUnit{Pica}{Ein Pica sind zwölf Punkt.}{pc}
   \TestUnit{Cicero}{Ein Cicero sind zwölf Didot-Punk.}{cc}
   \TestUnit{Zentimeter}{SI-Einheit}{cm}
   \TestUnit{Zoll (\textit{Inch})}{Ein Zoll entspricht \num{2.54}
      Zentimetern.}{in}
}

\providecommand{\RELATIVEUNITSHEADLINE}{Relative Einheiten}

\providecommand{\RELATIVEUNITSPRETEXT}{%
   Diese Einheiten sind abhängig von der jeweils verwendeten Schriftgröße.
   
   Details über die Einheiten Em und Ex unter: \url{http://tex.stackexchange.com/q/4239/4918}
   (Englisch)
}

\newcommand{\RELATIVEUNITSCONTENT}{%
   \TestUnit*
      {Geviert (\emph{Em})}
      {Traditionell wurde das Geviert als die Breite eines großen M definiert oder gleich
         der Schriftgröße gesetzt. Heutzutage wird der tatsächliche Wert in der Schriftdatei
         definiert.}
      [Diese Einheit sollte für alle horizontalen Abstände, die sich mit der Schriftgröße
         ändern, verwendet werden, z.\,B. den Absatzeinzug.]
      {em}
   \TestUnit*
      {x-Höhe (\emph{Ex})}
      {Diese Einheit bezog sich auf die Höhe eines kleinen x, wird heute aber wie das
         Gevierit in der Schriftdatei definiert.}
      [Diese Einheit sollte für alle vertikalen Abstände, die sich mit der Schriftgröße
         ändern, verwendet werden.]
      {ex}
   \TestUnit*
      {Matheeinheit (\textit{Math Unit})}
      {Eine Matheeinheit entspricht 1/18 Geviert der aktiven Mathesymbolschrift.}
      [Diese Einheit ist nur im Mathemodus verfügbar und kann dort nur für Abstände
         und Unterschneidungen eingesetzt werden.]
      {mu}
}

\newcommand{\INFOTEXT}{%
   Copyright \raisebox{-0.2ex}{©} 2016, Tobias Weh (\href{http://tobiw.de}{tobiw.de})\\
   Quellcode unter: \url{https://github.com/tweh/tex-units}
}
}{}
% However we load the english variable in every case to make sure to have the
% variables defined even in unkown languages:
% !TeX spellcheck = en_GB

\providecommand{\TITLE}{Length units in \TeX}

\providecommand{\INTRO}{%
   The given conversions may suffer form rounding errors and are only for guidance.
   
   The shown rules are \SI{1}{\mm} thick. Please note that the very short rules
   may not be rendered correctly on your screen (or printer) and appear longer
   than they are.
}

\providecommand{\ABSOLUTEUNITSHEADLINE}{Absolute Units}

\providecommand{\ABSOLUTEUNITSCONTENT}{%
   \TestUnit
      {Scaled Point}
      {The scaled point is defined as 1/\num{65536} points.}
      [This is the smallest unit \TeX\ uses.]
      {sp}
   \TestUnit{(PostScript-)Point}{The point is defined as 1/\num{72.27} inch.}{pt}
   \TestUnit
      {Big Point (DTP point)}
      {The big point is defined as 1/\num{72} Inch.}
      [Word, InDesign and other DTP applications use this definition for points.]
      {bp}
   \TestUnit{Didot Point}{An old unit used by European printers}{dd}
   \TestUnit{Millimeter}{SI unit}{mm}
   \TestUnit{Pica}{One pica equals twelve points.}{pc}
   \TestUnit{Cicero}{One Cicero equals twelve Didot points.}{cc}
   \TestUnit{Centimeter}{SI unit}{cm}
   \TestUnit{Inch}{One inch equals \num{2.54} centimeters.}{in}
}

\providecommand{\RELATIVEUNITSHEADLINE}{Relative Units}

\providecommand{\RELATIVEUNITSPRETEXT}{%
   These units depend on the currently active font size.
   
   For more details about em and ex see: \url{http://tex.stackexchange.com/q/4239/4918}
}

\providecommand{\RELATIVEUNITSCONTENT}{%
   \TestUnit*
      {Em}
      {Traditionally, an em was defined as the width of a capital M or to be equal to
         the font size, but today the actual value is defined in the font file.}
      [This unit should be used for all horizontal distances that should change relative
         to the font size; the paragraph indention for instance.]
      {em}
   \TestUnit*
      {Ex (x Height)}
      {Traditionally, an ex was defined as the height of a lower case x, but today
         the actual value is defined in the font file.}
      [This unit should be used for all vertical distances that should change relative
         to the font size.]
      {ex}
   \TestUnit*
      {Math Unit}
      {This unit equals approx 1/18 em of the math font family.}
      [It can only be used for spacing in math mode.]
      {mu}
}

\providecommand{\INFOTEXT}{%
   Copyright \raisebox{-0.2ex}{©} 2016, Tobias Weh (\href{http://tobiw.de/en}{tobiw.de/en})\\
   Source code available at: \url{https://github.com/tweh/tex-units}
}
% inside of this file we use \providecommand to define the variables so we don't
% overwrite the definitions of other languages.

\begin{document}
% CONTENT
{\Huge\bfseries\TITLE\par}
\vspace{\baselineskip}
\INTRO
\par\vspace{2\baselineskip}
\begin{multicols}{2}
   \PRIMARYUNITS
\end{multicols}
\par\vspace{2\baselineskip}
\begin{multicols}{3}
   \SECONDARYUNITS
\end{multicols}
\par\vfill
\footnotesize\itshape
\INFOTEXT
\par
\makebox[0pt][l]{\smash{
   \hspace{\RightColShift}%
   \begin{minipage}[t]{\RightColWidth}
      \vspace{-\textheight}
      \vspace{13mm}
      \hfill
      \DrawRules
   \end{minipage}
}}%
\makebox[0pt][l]{\smash{
   \hspace{\RightColShift}%
   \begin{minipage}[t]{\RightColWidth}
      \DrawRule{dd}{ptbarcolor}%
      \DrawRule{bp}{ptbarcolor}%
      \DrawRule{pt}{ptbarcolor}%
   \end{minipage}
}}%

% about em and ex see also http://tex.stackexchange.com/q/4239/4918.
% Donal Knuth wrote in his TeXBook (S. 60):
% > TeX also recognizes two units of measure that are relative rather than absolute;
% > i.e., they depend on the current context:
% >
% > em is the width of a “quad” in the current font;
% > ex is the “x-height” of the current font.
% >
% > Each font defines its own em and ex values. In olden days, an “em” was the width
% > of an ‘M’, but this is no longer true; ems are simply arbitrary units that come
% > with a font, and so are exes. The Computer Modern fonts have the property that
% > an em-dash is one em wide, each of the digits 0 to 9 is half an em wide, and
% > lowercase ‘x’ is one ex high; but these are not hard-and-fast rules for all
% > fonts. The \rm font (cmr10) of plain TeX has 1em = 10pt and 1ex ≈ 4.3pt; the \bf
% > font (cmbx10) has 1em = 11.5pt and 1ex ≈ 4.44pt; and the \tt font (cmtt10) has
% > 1em = 10.5pt and 1ex ≈ 4.3pt. All of these are “10-point” fonts, yet they have
% > different em and ex values. It is generally best to use em for horizontal
% > measurements and ex for vertical measurements that depend on the current font.

\end{document}